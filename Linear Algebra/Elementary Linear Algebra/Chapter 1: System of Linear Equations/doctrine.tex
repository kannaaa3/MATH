\documentclass{article}
\usepackage[utf8]{inputenc}
\usepackage{amsmath}
\usepackage{systeme}
\usepackage[most]{tcolorbox}

\usepackage[legalpaper,margin=1in]{geometry}

\setlength{\parindent}{10pt}
\setlength{\parskip}{1em}
\renewcommand{\baselinestretch}{1.2}

\title{SYSTEMS OF LINEAR EQUATIONS}
\date{}

\newcounter{example}[section]
\newenvironment{example}[1][]{\refstepcounter{example}\par\medskip
   \noindent \textbf{Example~\theexample. #1} \rmfamily}{\medskip}

\makeatletter
\renewcommand*\env@matrix[1][*\c@MaxMatrixCols c]{%
  \hskip -\arraycolsep
  \let\@ifnextchar\new@ifnextchar
  \array{#1}}
\makeatother

\newcommand\y{\cellcolor{blue!10}}
\newcommand\B{\textbf}

\usepackage{tabularray}
\SetTblrInner{colsep=5pt,rowsep=1pt}

\newcommand\x{\times}

\begin{document}
    \section{Introduction to Systems of Linear Equations}
    
    In general, a \textbf{linear equation} in n variables is defined as follows:
    \begin{tcolorbox}
        A \textbf{linear equation in n variables} $x_1,x_2,\dots,x_n$ has the form
        \begin{equation}
            a_1x_1 + a_2x_2 + ... + a_nx_n = b.
        \end{equation}
        \begin{itemize}
            \item         $a_1, a_2, \dots, a_n$ : \textbf{coefficients} (real nummbers)
            \item         $b$ :  \textbf{constant term} (real number)
        \end{itemize}
        
        $a_1$ : the \textbf{leading coefficients}\\
        $x_1$ : \textbf{leading variable}
    \end{tcolorbox}
    
    \textbf{Solution set} : The set of \textit{all solutions} of a linear equation.\\
    When this set is found, the equation is said to be \textbf{solved}. To describe the entire solution set
    of a linear equation, a \textbf{parametric representation} is used.

    \begin{example}
        Solve the linear equation $x_1 + 2x_2 = 4$.
        \begin{enumerate}
            \item Solve $x_1$ in terms of $x_2$, obtain 
                \[x_1 = 4 - 2x_2\]
                In this form, $x_2$ is \textbf{free} - it can take on any real value.\\
                $x_1$ is not free since its value depends on the value assigned to $x_2$.
            \item Represent the infinite number of solutions of this Eq, introduce a $3^{rd}$ var $t$ 
                called a \textbf{parameter}.

                $ \begin{cases}{}
                    x_1 = 4 - 2t\\
                    x_2 = t
                \end{cases}$
                , t is any real numbers.
        \end{enumerate}
    \end{example}

    \subsubsection*{Parametric Representation of a Solution Set}
    \begin{example}
        \[3x + 2y - z = 3\]
        Choosing $y$ and $z$ to be the free variables, obtain
        \[x = 1 - \frac{2}{3}y + \frac{1}{3}z\]
        Letting $y = s$ and $z = t$, obtain the parametric presentation
        \[x = 1 - \frac{2}{3}s + \frac{1}{3}t, \quad y = s, \quad z = t \]
        where $s$ and $t$ are any real numbers.
    \end{example}

    \subsection*{Systems of Linear Equations}

    A \textbf{system of $m$ linear equation in $n$ variables}:
    $  \begin{cases}{}
        a_{11}x_1 + a_{12}x_2 + \dots  + a_{1n}x_n = b_1 \\
        \vdots \\
        a_{m1}x_1 + a_{m2}x_2 + \dots  + a_{mm}x_n = b_m
    \end{cases}$

    A system of linear equations can have exactly one solution, an infinite number of solutions, or no solution.

\begin{itemize}
    \item \textbf{consistent} : $\geq 1$ solution
    \item \textbf{inconsistent} : no solution
\end{itemize}

    \subsection*{Solving a System of Linear Equations}

    \systeme {
        x - 2y + 3z = 9,
        y + 3z = 5,
        z = 2
    }

    This system is in \textbf{row-echelon form} (it follows a start-step pattern and has leading coefficients of 1). 
    To solve such a system, use a procedure called \textbf{back-substitution} (work backward).

    \subsection*{Gaussian Elimination}

    2 S.LN are called \textbf{equivalent}: have the same \textbf{solution set}.

    Changing the initial S.LN into an equivalent S.LN that is in  \textbf{row-echelon form}:
    \begin{enumerate}
        \item Interchange 2 equations
        \item Multiply an Eq. by a nonzero constant
        \item Add a multiple of an Eq. to another Eq.
    \end{enumerate}


    \section{Gaussian Elimination and Gauss-Jordan Elimination}
        
    This is a $m \times n$ matrix ($m$ by $n$ matrix) : 
    $\begin{bmatrix}
        a_{11} & a_{12} & \dots & a_{1n} \\
        a_{21} & a_{22} & \dots & a_{2n} \\
        \vdots & \vdots & & \vdots \\
        a_{m1} & a_{m2} & \dots & a_{mn}
    \end{bmatrix}$
    \begin{itemize}
        \item \textbf{entry} : $a_{ij}$
        \item \textbf{row subscript} : $i$
        \item \textbf{column subscript} : $j$
        \item \textbf{main diagonal} : the line that contains the entries $a_{11}, a_{22}, \dots$ (main diagonal entries)
    \end{itemize}
    
    $\begin{array}{ccc}
        \text{System} & \text{Augmented Matrix} & \text{ Coefficient Matrix} \\
        \systeme {
            x - 4y + 3z = 5,
            -x + 3y - z = -3,
            2x - 4z = 6
        }
        & \begin{bmatrix}[ccc|c]
            1  & -4 & 3 & 5\\
            -1 & 3 & -1 & -3\\
            2 & 0 & 4 & 6
        \end{bmatrix}
        & \begin{bmatrix}
            1 & -4 & 3\\
            -1 & 3 & -1\\
            2 & 0 & 4
        \end{bmatrix}
    \end{array}$

    \subsection*{Elementary Row Operations}
    \begin{enumerate}
        \item Interchange 2 Eq.
            ($R_1 \leftrightarrow R_2$)
        \item Multiply an Eq. by a nonzero constant. ($(\frac{1}{2}R_2) \rightarrow R_2$)
        \item Add a multiple of an Eq. to another Eq. ($R_3 + (-2)R_1 \to R_3$)
    \end{enumerate}

    \subsubsection*{Row-equivalent matrices}
    2 matrices are said to be \textbf{row-equivalent} if one can be obtained from the other by a finite sequence of 
    elementary row operations.
    
    \subsubsection*{Row-echelon form \& Reduce Row-echelon form}
    
    Matrices in row-echelon form
    \begin{equation*}\label{eq:appendrow}
        \left[\begin{tblr}{
        colspec = {cccc},
        cell{1}{1} = {blue9},
        cell{2}{1,2} = {blue9},
        cell{3}{1,2,3} = {blue9}}
        1 & 2 & -1 & 4 \\
        0 & 1 & 0 & 3 \\
        0 & 0 & 1 & -2
         \end{tblr}\right] 
         \quad
        \left[\begin{tblr}{
        cell{1}{1} = {blue9},
        cell{2}{1,2,3} = {blue9},
        cell{3}{1,2,3,4} = {blue9},
        row{4} = {blue9}}
            1 & -5 & 2 & -1 & 3\\
            0 & 0 & 1 & 3 & -3\\
            0 & 0 & 0 & 1 & 4\\
            0 & 0 & 0 & 0 & 1
         \end{tblr}\right]
    \end{equation*}

    Matrices in \textbf{reduced row-echelon form}
    \begin{equation*}
        \left[\begin{tblr}{
        colspec = {cccc},
        cell{1}{1,2} = {blue9},
        cell{2}{1,2,3} = {blue9},
        cell{3}{1,2,3,4} = {blue9}}
        0 & 1 & 0 & 5 \\
        0 & 0  & 1 & 3\\
        0 & 0 & 0 & 0
         \end{tblr}\right] 
    \end{equation*}

    \subsection*{Gauss-Jordan Elimination}

    Just the same but it continues the reduction until a \textit{reduced row-echelon} form is obtained.

    \begin{example}
        \systeme {
            x - 2y + 3z = 9,
            -x + 3y = -4,
            2x - 5y + 5z = 17
        } \quad
        $\begin{bmatrix}[ccc|c]
            1 & -2 & 3 & 9\\
            0 & 1 & 3 & -4\\
            2 & -5 & 5 & 17
        \end{bmatrix}$
        $\implies \begin{bmatrix}[ccc|c]
            1 & 0 & 0 & 1\\
            0 & 1 & 0 & -1\\
            0 & 0 & 1 & 2
        \end{bmatrix}$
    \end{example}

    \begin{tcolorbox}
        \textbf{What is better?}\\
        S.Eq are usually solved by computer. Most computer programs use a form of Gaussian elimination,
        with special emphasis on ways to reduce rounding errors and minimize storage of data.
    \end{tcolorbox}

    \section*{\textcolor{blue}{\textit{Homogeneous Systems of Linear Equations}}}
    S.Eq in which each of the \textbf{constant terms} is $0$ - such systems are called \textbf{homogeneous}.
    
    \begin{equation*}
        a_{11}x_1 + a_{12}x_2 + \dots + a_{1n}x_n = 0 \\
        \vdots \\
        a_{m1}x_1 + a_{m2}x_2 + \dots + a_{mn}x_n = 0
    \end{equation*}
    
    $\implies$ Definitely \textbf{consistent} (has at least 1 solution), that is 
    \begin{equation*}
        \text{\textbf{trivial (obvious) solution}}: x_1 = x_2 = \dots = x_n = 0
    \end{equation*}

    \begin{itemize}
        \item fewer Eq. than variables $\to$ infinite number of solution
        \item consistent
    \end{itemize}



    \iffalse
\begin{equation}\label{eq:appendrow}
  \left[\begin{tblr}{
    colspec = {cccc},
    row{1} = {red9},
    row{3} = {blue9},
    cell{5}{1,3} = {green9},
  }
    \x & \x & \x & \x \\
     0 & \x & \x & \x \\
     0 &  0 & \x & \x \\
     0 &  0 &  0 & \x \\
     a &  b &  c &  d \\
  \end{tblr}\right)
\end{equation}

\begin{equation}
  \left(\begin{tblr}{Q[c,olive9]cQ[c,yellow9]c}
    \x & \x & \x & \x \\
     0 & \x & \x & \x \\
     0 &  0 & \x & \x \\
     0 &  0 &  0 & \x \\
     a &  b &  c &  d \\
  \end{tblr}\right)
\end{equation}
    \fi
    



    \section{Applications of Systems of Linear Equations}

    \subsection{Polynomial Curve Fitting}
    
    Suppose a collection of data is represented by $n$ points in the $xy$-plane,\\
    \quad $(x_1,y_1), (x_2,y_2), \dots, (x_n, y_n)$

    Find a polynominal function of degree $n - 1$

    \quad $p(x) = a_0 + a_1x + \dots + a_{n-1}x^{n-1}$\\
    whose graph pass through the specified points.

    If all $x$-coordinates of the points are distinct, there is precisely 1 polynominal function
    of degree $n - 1$ (or less) that fits the $n$ points.

    Let $a_0, a_1, \dots, a_{n-1}$ be the n \textbf{variables} and substitute each of the $n$ points into the polynominal
    function 
    \begin{equation*}
        \begin{split}
            a_0 + a_1x_1 + \dots + a_{n-1}x_1^{n-1} & = y_1\\
            a_0 + a_1x_2 + \dots + a_{n-1}x_2^{n-1} & = y_2\\
       &   \vdots \\
            a_0 + a_1x_n + \dots + a_{n-1}x_n^{n-1} & =  y_n
        \end{split}
    \end{equation*}

    \subsubsection*{What if the $x$-values are large ?}

    \subsection*{Translating Large $x$-Values Before Curve Fitting}

    \begin{equation*}
        \overbrace{(2006, 3)}^{(x_1,y_1)}, \quad
        \overbrace{(2007, 5)}^{(x_2, y_2)}, \quad
        \overbrace{(2008, 1)}^{(x_3, y_3)}, \quad
        \overbrace{(2009, 4)}^{(x_4, y_4)}, \quad
        \overbrace{(2010, 10)}^{(x_5, y_5)}, \quad
    \end{equation*}
    
    Translation $z = x - 2008$ to obtain
    \begin{equation*}
        \overbrace{(-2, 3)}^{(z_1,y_1)}, \quad
        \overbrace{(-1, 5)}^{(z_2, y_2)}, \quad
        \overbrace{(0, 1)}^{(z_3, y_3)}, \quad
        \overbrace{(1, 4)}^{(z_4, y_4)}, \quad
        \overbrace{(2, 10)}^{(z_5, y_5)}, \quad
    \end{equation*}

    \[ \implies p(z) = 1 - \frac{5}{4}z  + \frac{101}{24}z^2 + \frac{3}{4}z^3 - \frac{17}{24}z^4.\]

    \begin{tcolorbox}
    \[ \implies p(=x) = 1 - \frac{5}{4}(x - 2008)  + \frac{101}{24}(x - 2008)^2 + \frac{3}{4}(x-2008)^3 - \frac{17}{24}(x - 2008)^4.\]
    \end{tcolorbox}

    \subsection*{Network Analysis}
    
    Network composed of \textbf{branches} and \textbf{junctions} - are used as models in economics, traffic analysis, and engineering.

    Assume in each of the junctions: \[ \sum \text{flow}_{in} = \sum \text{flow}_{out}\]
    Solve the linear equations for all junctions.
    
    \subsubsection*{Kirchhoff's Laws}
    \begin{enumerate}
        \item All the current flowing into a junction must flow out of it.
        \item The sum of $IR$  around a closed path is equal to the total voltage in the path.
    \end{enumerate}
    
    \section*{PROJECT}

    \subsection{Graphing Linear Equations}

    \subsection{Underdetermined and Overdetermined Systems of Equations}

    \begin{itemize}
        \item \textbf{underdetermined} : more \textbf{var} than \textbf{eq}
        \item \bf{overdetermined} : less \bf{var} than \bf{eq} 
    \end{itemize}




























\end{document}
