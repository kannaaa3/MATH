\documentclass{article}
\usepackage[utf8]{inputenc}
\usepackage{amsmath, amssymb, systeme, mathtools, lmodern, float, graphicx}
\usepackage[most]{tcolorbox}
\usepackage[scale=.95,type1]{cabin}
\usepackage[framemethod=tikz]{mdframed}

\usepackage[legalpaper,margin=0.2in]{geometry}

\setlength{\parindent}{10pt}
\setlength{\parskip}{1em}
\renewcommand{\baselinestretch}{1.2}

\title{Chapter 7: Eigenvalues and Eigenvectors}
\date{}

\makeatletter
\renewcommand*\env@matrix[1][*\c@MaxMatrixCols c]{%
  \hskip -\arraycolsep
  \let\@ifnextchar\new@ifnextchar
  \array{#1}}
\makeatother

\newcommand\y{\cellcolor{blue!10}}

\usepackage{tabularray}
\SetTblrInner{colsep=5pt,rowsep=1pt}

\newcommand\x{\times}
\newcommand\xor{\oplus}

\makeatletter
\newcommand{\dashover}[2][\mathop]{#1{\mathpalette\df@over{{\dashfill}{#2}}}}
\newcommand{\fillover}[2][\mathop]{#1{\mathpalette\df@over{{\solidfill}{#2}}}}
\newcommand{\df@over}[2]{\df@@over#1#2}
\newcommand\df@@over[3]{%
  \vbox{
    \offinterlineskip
    \ialign{##\cr
      #2{#1}\cr
      \noalign{\kern1pt}
      $\m@th#1#3$\cr
    }
  }%
}
\newcommand{\dashfill}[1]{%
  \kern-.5pt
  \xleaders\hbox{\kern.5pt\vrule height.4pt width \dash@width{#1}\kern.5pt}\hfill
  \kern-.5pt
}
\newcommand{\dash@width}[1]{%
  \ifx#1\displaystyle
    2pt
  \else
    \ifx#1\textstyle
      1.5pt
    \else
      \ifx#1\scriptstyle
        1.25pt
      \else
        \ifx#1\scriptscriptstyle
          1pt
        \fi
      \fi
    \fi
  \fi
}
\newcommand{\solidfill}[1]{\leaders\hrule\hfill}
\makeatother

\newcommand\R{\mathbb{R}}

\DeclarePairedDelimiter\abs{\lvert}{\rvert}%
\DeclarePairedDelimiter\norm{\lVert}{\rVert}%

% Swap the definition of \abs* and \norm*, so that \abs
% and \norm resizes the size of the brackets, and the 
% starred version does not.
\makeatletter
\let\oldabs\abs
\def\abs{\@ifstar{\oldabs}{\oldabs*}}
%
\let\oldnorm\norm
\def\norm{\@ifstar{\oldnorm}{\oldnorm*}}
\makeatother

\newcommand*{\Value}{\frac{1}{2}x^2}%

\newcommand\ddfrac[2]{\frac{\displaystyle #1}{\displaystyle #2}}


\newcounter{Ques}[section]
\newenvironment{Ques}[1][]{%
    \stepcounter{Ques}
  \ifstrempty{#1}%
  {\mdfsetup{%
    frametitle={%
      \tikz[baseline=(current bounding box.east),outer sep=0pt]
      \node[line width=1pt,anchor=east,rectangle,draw=blue!20,fill=white]
    {\strut \color{blue5}{\textit{QUESTION}}~\theQues};}}
  }%
  {\mdfsetup{%
    frametitle={%
      \tikz[baseline=(current bounding box.east),outer sep=0pt]
      \node[line width=1pt,anchor=east,rectangle,draw=blue!20,fill=white]
    {\strut \color{black}{\textit{ANSWER}}~:~\color{blue4}{#1}};}}%
  }%
  \mdfsetup{innertopmargin=2pt,linecolor=blue!20,%
            linewidth=1pt,topline=true,%
            frametitleaboveskip=\dimexpr-\ht\strutbox\relax,}
  \begin{mdframed}[]\relax%
  }{\end{mdframed}}


\begin{document}
    \begin{Ques}
        Determine whether the matrix is symmetric.
    \end{Ques}
    \begin{minipage}{0.20\linewidth}
        \[A = \begin{bmatrix}[rr]
            6 & -2 \\
            -2 & 1 
        \end{bmatrix} \]
        
    \end{minipage}
    \begin{minipage}{0.20\linewidth}
        \[A = \begin{bmatrix}[rr]
            1 & 3 \\
            2 & 4 
        \end{bmatrix} \]
    \end{minipage}
    \begin{minipage}{0.20\linewidth}
        \[A = \begin{bmatrix}[rrr]
            2 & -2 &  1 \\
            -2 & 3 & 4 \\
            0 & 4 & 1 
        \end{bmatrix} \]

    \end{minipage}
    \begin{minipage}{0.20\linewidth}
        \[A = \begin{bmatrix}[rrr]
            -5 & 3 & 4 \\
            3 & 7 & -2 \\
            4 & -2 & 3
        \end{bmatrix} \]

    \end{minipage}

    \begin{minipage}{0.20\linewidth}
        (a) Yes.
    \end{minipage}
    \begin{minipage}{0.20\linewidth}
        (b) No.
    \end{minipage}
    \begin{minipage}{0.20\linewidth}
        (c) No.
    \end{minipage}
    \begin{minipage}{0.20\linewidth}
        (d) Yes.
    \end{minipage}


    \begin{Ques}
             Find the eigenvalues with their eigenvectors and the dimensions of the corresponding eigenspaces.

    \end{Ques}

    \begin{minipage}{0.48\linewidth}
    (a) $A = \begin{bmatrix}[rr]
        1 & 3 \\
        3 & 1
    \end{bmatrix} $
   
    $            | \lambda I - A| = \begin{vmatrix}
                \lambda  - 1 & -3 \\
                -3 & \lambda  - 1
            \end{vmatrix} = ( \lambda  - 4 )( \lambda  + 2 )$
    
    $\bullet$ For $ \lambda _1 = 4$: $ 4I - A = \begin{bmatrix}[rr]
        3 & -3 \\
        -3 & 3
    \end{bmatrix} \rightarrow \begin{bmatrix}[rr]
        1 & -1 \\
        0 & 0 
    \end{bmatrix} $

    Let $s = x_2 \implies \textbf{x} = \begin{bmatrix}[r]
        x_1 \\
        x_2
    \end{bmatrix} = s \begin{bmatrix}[c]
        1 \\ 1
    \end{bmatrix}$

    $\implies \textbf{u}_1 = \begin{bmatrix}[r]
        1 \\
        1 
    \end{bmatrix} \quad \text{(dim = 1)}$


    $\bullet$ For $ \lambda _2 = -2$: $-2I - A = \begin{bmatrix}[rr]
        -3 & -3 \\
        -3 & -3 
    \end{bmatrix} \to \begin{bmatrix}[rr]
        1 & 1 \\
        0 & 0 
    \end{bmatrix} $

    Let $s = x_2 \implies \textbf{x} = \begin{bmatrix}[c]
        x_1 \\ x_2 
    \end{bmatrix} = \begin{bmatrix}[r]
        -s \\ s 
    \end{bmatrix} = s \begin{bmatrix}[r]
        -1 \\ 1 
    \end{bmatrix} $

    $\implies \textbf{u}_2 = \begin{bmatrix}[r]
        -1 \\
        1 
    \end{bmatrix} \quad \text{(dim = 1)}$

    \end{minipage} \hfill
    \begin{minipage}{0.48\linewidth}
        (b) $A = \begin{bmatrix}[rr]
            0 & 2 \\
            2 & 0 
        \end{bmatrix} $

    $| \lambda I - A| = \begin{vmatrix}
        \lambda  & -2 \\
        -2 & \lambda  
    \end{vmatrix}$

    Let $\lambda = \ddfrac{-2(\lambda' - 1)}{-3} $, the characteristic polynomial has been solved for $\lambda'$ in $(a)$.
    
    $\bullet$ For $ \lambda _1 = \ddfrac{2(\lambda_1' - 1)}{3} = 2, \quad \textbf{u}_1 = \begin{bmatrix}[r]
        1 \\
        1 
    \end{bmatrix} \quad \text{(dim = 1)}$


    $\bullet$ For $ \lambda _2 = \ddfrac{2(\lambda_2' - 1)}{3} = -2, \quad \textbf{u}_2 = \begin{bmatrix}[r]
        -1 \\
        1 
    \end{bmatrix} \quad \text{(dim = 1)}$


    \end{minipage}

    {\color{blue9} \rule{20cm}{0.3mm}}

    

%############################################################################################################

    \begin{minipage}[t]{0.48\linewidth}
        (c) $A = \begin{bmatrix}[rrr]
            2 & 1 & 1 \\
            1 & 2 & 1 \\
            1 & 1 & 2 
        \end{bmatrix} $

        $| \lambda I - A| = \begin{vmatrix}
            \lambda  - 2 & -1 & -1 \\
            -1 & \lambda  - 2 & -1 \\
            -1 & -1 & \lambda  - 2
        \end{vmatrix} $

        Let $ \lambda ' = \lambda  - 2$. The determinant become the one in $(f)$, we already solved for $ \lambda '$.
        
        $\bullet$ For $\lambda_1 = \lambda_1' + 2 = 1, \quad \begin{cases}{}
            \textbf{u}_1 = (-1,1,0) \\
            \textbf{u}_2 = (-1,0,1)
        \end{cases} \quad \text{(dim = 2)}$

        $\bullet$ For $\lambda_2 = \lambda_2' + 2 = 4, \quad \textbf{u}_3 = (1,1,1) \quad \text{(dim = 1)}$

    \end{minipage} \hfill
    \begin{minipage}[t]{0.48\linewidth}
        (d) $A = \begin{bmatrix}[rrr]
            0 & 2 & 2 \\
            2 & 0 & 2 \\
            2 & 2 & 0 
        \end{bmatrix} $

        $| \lambda I - A| = \begin{vmatrix}
            \lambda  & -2 & -2 \\
            -2 & \lambda  & -2 \\
            -2 & -2 & \lambda  
        \end{vmatrix} = ( \lambda  + 2 )^2( \lambda  - 4 )$

        $\bullet$ For $ \lambda _1 = -2$: $2I - A = \begin{bmatrix}[rrr]
            -2 & -2 & -2 \\
            -2 & -2 & -2 \\
            -2 & -2 & -2
        \end{bmatrix} \to \begin{bmatrix}[rrr]
            1 & 1 & 1 \\
            0 & 0 & 0 \\
            0 & 0 & 0
        \end{bmatrix}  $

        Let $s = x_2$ and $t = x_3$: $ \textbf{x} = s \begin{bmatrix}[r]
            -1 \\
            1 \\ 0 
        \end{bmatrix} + t \begin{bmatrix}[r]
            -1 \\
            0 \\
            1 
        \end{bmatrix} $

        $\implies \begin{cases}{}
            \textbf{u}_1 = (-1, 1, 0) \\
            \textbf{u}_2 = (-1, 0, 1)
        \end{cases} \quad \text{(dim = 2)}$


    $\bullet$ For $ \lambda _2 = 4: 4I - A = \begin{bmatrix}[rrr]
        4 & -2 & -2 \\
        -2 & 4 & -2 \\
        -2 & -2 & 4 
    \end{bmatrix} \to \begin{bmatrix}[rrr]
        1 & 0 & -1 \\
        0 & 1 & -1 \\
        0 & 0 & 0 
    \end{bmatrix} $

    Let $s = x_3: \textbf{x} = s \begin{bmatrix}[r]
        1 \\ 1 \\ 1 
    \end{bmatrix}$

    $\implies \textbf{u}_3 = (1,1,1) \quad \text{(dim = 1)}$


    \end{minipage}

    \begin{minipage}[t]{0.48\linewidth}
        (e) $A = \begin{bmatrix}[rrr]
            0 & 4 & 4 \\
            4 & 2 & 0 \\
            4 & 0 & -2 
        \end{bmatrix}$

        $| \lambda I - A| = \begin{vmatrix}
            \lambda  & -4 & -4 \\
            -4 & \lambda  -2 & 0 \\
            -4 & 0 & \lambda  + 2 
        \end{vmatrix} = \lambda ( \lambda  - 6 )( \lambda  + 6 )$

        $\bullet$  For $ \lambda _1 = 0 : 0I - A = \begin{bmatrix}[rrr]
            0 & -4 & - 4 \\
            -4 & -2 & 0 \\
            -4 & 0 & 2 
        \end{bmatrix} \to \begin{bmatrix}[rrr]
            2 & 0 & -1 \\ 
            0 & 1 & 1 \\
            0 & 0 & 0 
        \end{bmatrix} $

        Let $s = x_3: \textbf{x} = s \begin{bmatrix}[r]
            1/2 \\
            -1 \\
            1
        \end{bmatrix} \implies \textbf{w} _1 = ( \ddfrac{1}{2} , -1, 1 ) \quad \text{(dim = 1)}$

        $\bullet$ For $ \lambda _2 = 6: 6I - A = \begin{bmatrix}[rrr]
            6 & -4 & - 4 \\
            -4 & 4 & 0 \\
            -4 & 0 & 8 
        \end{bmatrix} \to \begin{bmatrix}[rrr]
            1 & 0 & -2 \\
            0 & 1 & -2 \\
            0 & 0 & 0 
        \end{bmatrix} $

        Let $s = x_3: \textbf{x} = s \begin{bmatrix}[r]
            2 \\
            2 \\
            1 
        \end{bmatrix} \implies \textbf{w} _2 = (2, 2, 1) \quad \text{(dim = 1)}$
        
        $\bullet$ For $ \lambda _3 = -6: -6I - A = \begin{bmatrix}[rrr]
            -6 & -4 & -4 \\
            -4 & -8 & 0 \\
            -4 & 0 & -4 
        \end{bmatrix} \to \begin{bmatrix}[rrr]
            1 & 0 & 1 \\
            0 & 2 & -1 \\
            0 & 0 & 0 
        \end{bmatrix} $

        Let $s = x_3: \textbf{x} = s \begin{bmatrix}[r]
            -1 \\
            1/2 \\
            1 
        \end{bmatrix} \implies \textbf{w} _3 = (-1, \ddfrac{1}{2} , 1) \quad \text{(dim = 1)}$



    \end{minipage} \hfill 
    \begin{minipage}[t]{0.48\linewidth}
        (f) $A_{(f)} = \begin{bmatrix}[rrr]
            0 & 1 & 1 \\
            1 & 0 & 1 \\
            1 & 1 & 0 
        \end{bmatrix} = \ddfrac{1}{2} A_{(d)}  \implies  A_{(d)} \textbf{x} =2 \lambda  x$

        This implies that the eigenvalues of $A_{(f)}$ are equal to half of the ones from $(d)$ with
        the corresponding eigenvectors.

        $\bullet \lambda _1 = \frac{1}{2} (-2) = -1, \quad \begin{cases}{}
            \textbf{u}_1 = (-1,1,0) \\
            \textbf{u}_2 = (-1, 0, 1)
        \end{cases} \quad \text{(dim = 2)}$

        $\bullet \lambda _2 = \frac{1}{2}(4) = 2,\quad \textbf{u}_3 = (1,1,1) \quad \text{(dim = 1)}$

    \end{minipage}

    \begin{minipage}[t]{0.48\linewidth}
        (g) $A = \begin{bmatrix}[rrr]
            2 & -1 & -1 \\
            -1 & 2 &  -1\\
            -1 & -1 & 2
        \end{bmatrix} $
        
        $| \lambda I - A| = \begin{vmatrix}
            \lambda  - 2 & 1 & 1 \\
            1 & \lambda  - 2 & 1 \\
            1 & 1 & \lambda  - 2 
        \end{vmatrix} $

        Let $\lambda' = 2 - \lambda$, then we already solved for $\lambda'$ in $(f)$.

        $\bullet$ For $ \lambda _1 = 2 - \lambda_1' = 3, \quad 
         \begin{cases}{}
            \textbf{u}_1  = (-1, 1, 0) \\
            \textbf{u} _2 = (-1, 0, 1)
        \end{cases} \quad \text{(dim = 2)} $

        $\bullet$ For $ \lambda _2 = 2 - \lambda_2' = 0, \quad \textbf{u} _3 = (1,1,1) \quad \text{(dim = 1)}$



    \end{minipage} \hfill
    \begin{minipage}[t]{0.48\linewidth}
        (h) $A = \begin{bmatrix}[rrr]
            3 & 0 & 0 \\
            0 & 1 & 0 \\
            0 & 0 & 1 
        \end{bmatrix}$

        $\bullet$ For $ \lambda _1 = 3$: the dimension of the eigenspace of $ \lambda _1$ is 1.

        $3I - A = \begin{bmatrix}[rrr]
            0 & 0 & 0 \\
            0 & 2 & 0 \\
            0 & 0 & 2 
        \end{bmatrix} \to \begin{bmatrix}[rrr]
            0& 1 & 0 \\
            0 & 0 & 1 \\
            0 & 0 & 0 
        \end{bmatrix} $

        Let $s = x_1 : \textbf{x} = s \begin{bmatrix}[r]
            1 \\
            0 \\
            0 
        \end{bmatrix} \implies \textbf{u}_1 = (1,0,0)$

        $\bullet$ For $ \lambda _2 = 1$: since $ \lambda _2$ has a multiplicity of $2$, the corresponding eigenspace has dimesion $2$.

        $I - A = \begin{bmatrix}[rrr]
            -2 & 0 & 0 \\
            0 & 0 & 0 \\
            0 & 0 & 0 
        \end{bmatrix} \to \begin{bmatrix}[rrr]
            1 & 0 & 0 \\
             0  &0 &0 \\ 0& 0 & 0 
        \end{bmatrix} $

        Let $s = x_2$ and $t = x_3 : \textbf{x} = s \begin{bmatrix}[r]
            0\\
            1 \\
            0 
        \end{bmatrix} + t \begin{bmatrix}[r]
            0 \\
            0 \\
            1 
        \end{bmatrix} \implies \begin{cases}{}
            \textbf{u}_2 = (0,1,0) \\
            \textbf{u}_3 = (0,0,1)
        \end{cases}$
    \end{minipage}


    \pagebreak
    %%%%%%%%%%%%%%%%%%%%%%%%%%%%%%%%%%%%%%%%%%%%%%%%%%%%%%%%%%%%%%%%%%%%%%%%%%%%%%%%%%%%%%%%%%%%%%%%%%%%%%%%%%%%%%

    \begin{Ques}
            Determine whether the matrix is \textbf{orthogonal}.

    \end{Ques}


    \begin{minipage}{0.45\linewidth}
        \[ \begin{bmatrix}[rr]
            \ddfrac{\sqrt{2}}{2}  & \ddfrac{\sqrt{2}}{2} \\
            -\ddfrac{\sqrt{2}}{2} & \ddfrac{\sqrt{2}}{2} 
        \end{bmatrix} \]
        (a) Yes.

        The column vectors of $A$ form an orthonormal basis.
    \end{minipage}
    \begin{minipage}{0.45\linewidth}
        \[ \begin{bmatrix}[rr]
            2/3 & -2/3 \\
            2/3 & 1/3 
        \end{bmatrix} \]
        (b) No.
        \[ \norm{a_1} = \ddfrac{2\sqrt{2}}{3} \ne 1\]
    
    \end{minipage}

    \begin{minipage}{0.45\linewidth}
        \[A = \begin{bmatrix}[rrr]
            -4 & 0 & 3 \\
            0 & 1 & 0 \\
            3 & 0 & 4 
        \end{bmatrix} \]
        (c) No.
        \[\norm{a_1} = 5 \ne 0 \]
    \end{minipage}
    \begin{minipage}{0.45\linewidth}
        \[A = \begin{bmatrix}[rrr]
            -4/5 & 0 & 3/5 \\
            0 & 1 & 0 \\
            3/5 & 0 & 4/5 
        \end{bmatrix} \]
        (d) Yes.

        The column vectors of $A$ form an orthonormal basis.
    \end{minipage}

    {\color{blue9} \rule{10cm}{0.3mm}} 

    \begin{Ques}
             Find an orthogonal matrix $P$ that diagonalizes $A$ and $P^TAP$.
    \end{Ques}


    \begin{minipage}{0.499\linewidth}
        (a) $A = \begin{bmatrix}
            1 & 1 \\
            1 & 1 
        \end{bmatrix} $

        $| \lambda  I - A | = \begin{vmatrix}
            \lambda  - 1 & - 1 \\
            -1 & \lambda  - 1 
        \end{vmatrix} = ( \lambda  - 2 ) \lambda  $

        $\bullet$ For $ \lambda _1 = 2$: $2I - A = \begin{bmatrix}[rr]
            1 & -1 \\
            -1 & 1 
        \end{bmatrix} \to \begin{bmatrix}[rr]
            1 & -1 \\
            0 & 0 
        \end{bmatrix} $

        Let $s = x_2 \implies \textbf{x} = \begin{bmatrix}[r]
            s \\
            s
        \end{bmatrix}  = s \begin{bmatrix}[r]
            1 \\
            1
        \end{bmatrix} \implies \textbf{u}_1 = \left( \ddfrac{1}{\sqrt{2}} ,
        \ddfrac{1}{\sqrt{2}} \right)$

        $\bullet$ For $ \lambda _2 = 0$: $0I - A = \begin{bmatrix}[rr]
            -1 & -1\\
            -1 & -1 
        \end{bmatrix} \to \begin{bmatrix}[rr]
            1 & 1 \\
            0 & 0 
        \end{bmatrix} $

        Let $s = x_2 \implies \textbf{x} = \begin{bmatrix}[r]
            -s \\
            s
        \end{bmatrix} = s \begin{bmatrix}[r]
            -1 \\
            1 
        \end{bmatrix} \implies \textbf{u}_2 = \left( - \ddfrac{1}{\sqrt{2}} , \ddfrac{1}{\sqrt{2}}  \right)$
        \begin{equation*}
            \begin{split}
                P &= \begin{bmatrix}[rr]
            \frac{1}{\sqrt{2}} & -\frac{1}{\sqrt{2}} \\
            \frac{1}{\sqrt{2}} & \frac{1}{\sqrt{2}}
        \end{bmatrix} \\
                    P^TAP &= \begin{bmatrix}[rr]
                        2 & 0 \\
                        0 & 0
                    \end{bmatrix} 
            \end{split}
        \end{equation*}
    \end{minipage}
    \begin{minipage}{0.45\linewidth}
        (b) $A = \begin{bmatrix}[rr]
            4 & 2 \\
            2 & 4 
        \end{bmatrix} $

        $| \lambda I - A | = \begin{vmatrix}
            \lambda  - 4 & 2 \\
            2 & \lambda  - 4 
        \end{vmatrix} =  ( \lambda  - 6 )( \lambda  - 2 )$

        $\bullet$ For $ \lambda _1 = 6$: $6I - A = \begin{bmatrix}[rr]
            2 & 2 \\
            2 & 2 
        \end{bmatrix} \to \begin{bmatrix}[rr]
            1 & 1 \\
            0 & 0 
        \end{bmatrix} $

        Let $s = x_2: \textbf{x} = s \begin{bmatrix}[r]
            -1 \\
            1 
        \end{bmatrix} \implies \textbf{u}_1 = \left( - \ddfrac{1}{\sqrt{2}} , \ddfrac{1}{\sqrt{2}}  \right) $

        $\bullet$ For $ \lambda _2 = 2$: $2I - A = \begin{bmatrix}[rr]
            -2 & 2 \\
            2 & -2 
        \end{bmatrix} \to \begin{bmatrix}[rr]
            1 & -1 \\
            0 & 0 
        \end{bmatrix} $

        Let $s = x_2: \textbf{x} = s \begin{bmatrix}[r]
            1 \\
            1 
        \end{bmatrix} \implies \textbf{u}_2 = \left( \ddfrac{1}{\sqrt{2}} , \ddfrac{1}{\sqrt{2}}  \right) $
        \begin{equation*}
            \begin{split}
                P &= \begin{bmatrix}[rr]
                    -\ddfrac{1}{\sqrt{2}}  & \ddfrac{1}{\sqrt{2}}  \\
                    \ddfrac{1}{\sqrt{2}}  & \ddfrac{1}{\sqrt{2}}  
                \end{bmatrix} \\
                    P^TAP& = \begin{bmatrix}[rr]
                    6 & 0 \\
                    0 & 2
                \end{bmatrix} 
            \end{split}
        \end{equation*}
    \end{minipage}

    {\color{blue9} \rule{20cm}{0.3mm}}

    

    \begin{minipage}{0.54\linewidth}
        (c) $A = \begin{bmatrix}[ccc]
            0 & 3 & 0 \\
            3 & 0 & 4 \\
            0 & 4 & 0 
        \end{bmatrix} $

        $| \lambda I - A| = \begin{vmatrix}
            \lambda  & -3 & 0 \\
            -3 & \lambda  & -4 \\
            0 & -4 & \lambda  
        \end{vmatrix} = \lambda ( \lambda  - 5 )( \lambda  + 5 )$

        $\bullet$ For $ \lambda _1 = 0$: $0I - A = \begin{bmatrix}[rrr]
            0 & -3 & 0 \\
            -3 & 0 & -4 \\
            0 & -4 & 0 
        \end{bmatrix} \to \begin{bmatrix}[rrr]
            3 & 0 & 4 \\
            0 & 1 & 0 \\
            0 & 0 & 0 
        \end{bmatrix} $

        Let $s = x_3: \textbf{x} = \begin{bmatrix}[r]
            -\frac{4}{3}s \\
            0 \\
            s 
        \end{bmatrix} = s \begin{bmatrix}[r]
            - \frac{4}{3} \\
            0 \\
            1 
        \end{bmatrix} \implies \textbf{u}_1 = \left( - \frac{4}{5}, 0, \frac{3}{5} \right)$

        $\bullet$ For $ \lambda _2 = 5$: $5I - A = \begin{bmatrix}[rrr]
            5 & -3 & 0 \\
            -3 & 5 & -4\\
            0 & -4 & 5 
        \end{bmatrix} \to \begin{bmatrix}[rrr]
            4 & 0 & -3 \\
            0 & 4 & -5 \\
            0 & 0 & 0 
        \end{bmatrix}$
        
        Let $s = x_3 : \textbf{x} = \begin{bmatrix}[r]
            \frac{3}{4}s \\
            \frac{5}{4}s \\
            s
        \end{bmatrix} =s \begin{bmatrix}[r]
            \frac{3}{4} \\
            \frac{5}{4} \\
            1 
        \end{bmatrix} \implies \textbf{u}_2 = \left( \ddfrac{3}{5\sqrt{2}} , \ddfrac{1}{\sqrt{2}} , \ddfrac{4}{5\sqrt{2}}  \right) $ 

        $\bullet$ For $ \lambda _3 = -5$: $5I - A = \begin{bmatrix}[rrr]
            -5 & -3 & 0 \\
            -3 & -5 & -4 \\
            0 & -4 & -5 
        \end{bmatrix} \to \begin{bmatrix}[rrr]
            4 & 0 & -3 \\
            0 & 4 & 5 \\
            0 & 0 & 0 
        \end{bmatrix} $

        Let $s = x_3 : \textbf{x} = \begin{bmatrix}[r]
            \frac{3}{4}s \\
            - \frac{5}{4}s \\
            s 
        \end{bmatrix} = s \begin{bmatrix}[r]
            \frac{3}{4} \\
            - \frac{5}{4}\\
            1
        \end{bmatrix} \implies \textbf{u}_3 = \left( \ddfrac{3}{5\sqrt{2}} , -\ddfrac{1}{\sqrt{2}} , \ddfrac{4}{5\sqrt{2}}  \right)$
        \begin{equation*}
            \begin{split}
                P &= \begin{bmatrix}[rrr]
            -\ddfrac{4}{5} & \ddfrac{3}{5\sqrt{2}} & \ddfrac{3}{5\sqrt{2}} \\
            0 & \ddfrac{1}{\sqrt{2}} & - \ddfrac{1}{\sqrt{2}} \\
            \ddfrac{3}{5} &  \ddfrac{4}{5\sqrt{2}} & \ddfrac{4}{5\sqrt{2}} 
        \end{bmatrix} \\
                    P^TAP &= \begin{bmatrix}[rrr]
                        0 & 0 & 0 \\
                        0 & 5 & 0 \\
                        0 & 0 & -5 
                    \end{bmatrix} 
            \end{split}
        \end{equation*}
    \end{minipage} %$$$$$$$$$$$$$$$$$$$$$$$$$$$$$$$$$$$$$$$$$$$$$$$$$$$$$$$$$$$$$$$$$$$$$$$$$$$$$$$$$
    \begin{minipage}{0.43\linewidth}
        (d) $A = \begin{bmatrix}[rrr]
            1 & -1 & 2 \\
            -1 & 1 & 2 \\
            2 & 2 & 2 
        \end{bmatrix} $

        $| \lambda I - A| = \begin{vmatrix}
            \lambda  - 1 & 1 & -2 \\
            1 & \lambda  - 1 & -2 \\
            -2 & -2 & \lambda  - 2 
        \end{vmatrix} = ( \lambda  - 4 )( \lambda  - 2 )( \lambda  + 2 )$

        $\bullet$ For $ \lambda _1 = 4$: $4I - A = \begin{bmatrix}[rrr]
            3 & 1 & -2 \\
            1 & 3 & -2 \\
            -2 & -2 & 2 
        \end{bmatrix} \to \begin{bmatrix}[rrr]
            2 & 0 & -1 \\
            0 & 2 & -1 \\
            0 & 0 & 0 
        \end{bmatrix}$

        Let $s = x_3: \textbf{x} = s \begin{bmatrix}[r]
            \frac{1}{2} \\
            \frac{1}{2}  \\
            1 
        \end{bmatrix} \implies \textbf{u}_1 = \left( \ddfrac{1}{\sqrt{6}} , \ddfrac{1}{\sqrt{6}} , \ddfrac{\sqrt{2}}{\sqrt{3}}  \right) $

        $\bullet$ For $ \lambda _2 = 2$: $2I - A = \begin{bmatrix}[rrr]
            1 & 1 & -2 \\
            1 & 1 & -2 \\
            -2 & -2 & 0 
        \end{bmatrix} \to \begin{bmatrix}[rrr]
            1 & 1 & 0 \\
            0 & 0 & 2 \\
            0 & 0 & 0 
        \end{bmatrix} $

        Let $s = x_2: \textbf{x} = s \begin{bmatrix}[r]
            -1 \\
            1 \\
            0 
        \end{bmatrix}  \implies \textbf{u}_2 = \left( - \ddfrac{1}{\sqrt{2}} , \ddfrac{1}{\sqrt{2}} , 0 \right) $

        $\bullet$ For $ \lambda _3 = -2$: $-2I - A =  \begin{bmatrix}[rrr]
            -3 & 1 & -2 \\
            1 & -3 & -2 \\
            -2 & -2 & -4 
        \end{bmatrix} \to \begin{bmatrix}[rrr]
            1 & 0 & 1 \\
            0 & 1 & 1 \\
            0 & 0 & 0 
        \end{bmatrix} $

        Let $s = x_3: \textbf{x} = s \begin{bmatrix}[r]
            -1 \\
            -1 \\
            1 
        \end{bmatrix} \implies \textbf{u}_3 = \left( - \ddfrac{1}{\sqrt{3}} , - \ddfrac{1}{\sqrt{3}} , \ddfrac{1}{\sqrt{3}}  \right)$
        \begin{equation*}
            \begin{split}
                P &= \begin{bmatrix}[rrr]
                    \ddfrac{1}{\sqrt{6}} & - \ddfrac{1}{\sqrt{2}}  & - \ddfrac{1}{\sqrt{3}} \\
                    \ddfrac{1}{\sqrt{6}} & \ddfrac{1}{\sqrt{2}} & - \ddfrac{1}{\sqrt{2}} \\
                    \ddfrac{\sqrt{2}}{\sqrt{3}}  & 0 & \ddfrac{1}{\sqrt{3}} 
                \end{bmatrix}  \\
                    P^TAP& = \begin{bmatrix}[rrr]
                    4 & 0 & 0 \\
                    0 & 2 & 0 \\
                    0 & 0 & -2
                \end{bmatrix} 
            \end{split}
        \end{equation*}
    \end{minipage}%$$$$$$$$$$$$$$$$$$$$$$$$$$$$$$$$$$$$$$$$$$$$$$$$$$$$$$$$$$$$$$$$$$$$$$$$$$$$$$$$$$
    
    {\color{blue8} \rule{20cm}{0.3mm}}

    (e)  $A = \begin{bmatrix}[cccc]
            1 & 1 & 0 & 0\\
            1 & 1 & 0 & 0 \\
            0 & 0 & 1 & 1 \\
            0 & 0 & 1 & 1 
        \end{bmatrix} \implies | \lambda I - A| = \begin{vmatrix}
            \lambda  - 1 & -1 & 0 & 0 \\
            -1 & \lambda  - 1 & 0 & 0 \\
            0 & 0 & \lambda  - 1 & -1 \\
            0 & 0 & -1 & \lambda  - 1
        \end{vmatrix} = \left[ ( \lambda  - 1 )^2 - 1 \right]^2 = ( \lambda  - 2 )^2 \lambda ^2  $ 
        
        $\bullet$ For $ \lambda _1 = 2$: $2I - A = \begin{bmatrix}[rrrr]
            1 & -1 & 0 & 0 \\
            -1 & 1 & 0 & 0 \\
            0 & 0 & 1 & -1 \\
            0 & 0 & -1 & 1 
        \end{bmatrix} \to \begin{bmatrix}[rrrr]
            1 & -1 & 0 & 0 \\
            0 & 0 & 1 & -1 \\
            0 & 0 & 0 & 0 \\
            0 & 0 & 0 & 0 
        \end{bmatrix} $. 

        Let $s = x_2$ and $t = x_4$: $ \textbf{x} = s \begin{bmatrix}[r]
            1 \\
            1 \\
            0 \\
            0 
        \end{bmatrix} + t \begin{bmatrix}[r]
            0 \\
            0 \\
            1 \\
            1 
        \end{bmatrix} \implies \begin{cases}{}
            \textbf{u}_1 = \left( \ddfrac{1}{\sqrt{2}} , \ddfrac{1}{\sqrt{2}} , 0, 0 \right) \\
            \textbf{u}_2 = \left( 0, 0, \ddfrac{1}{\sqrt{2}} , \ddfrac{1}{\sqrt{2}}  \right) 
        \end{cases}$

        $\bullet$ For $ \lambda _2 = 0$: $0I - A = \begin{bmatrix}[rrrr]
            -1 & -1 & 0 & 0 \\
            -1 & -1 & 0 & 0 \\
            0 & 0 & -1 & -1 \\
            0 & 0 & -1 & -1 
        \end{bmatrix} \to \begin{bmatrix}[rrrr]
            1 & 1 & 0 & 0 \\
            0 & 0 & 1 & 1 \\
            0 & 0 & 0 & 0 \\
            0 & 0 & 0 & 0 
        \end{bmatrix} $.

        Let $s = x_2$ and $t = x_4$: $ \textbf{x} = s \begin{bmatrix}[r]
            -1 \\
            1 \\
            0 \\
            0 
        \end{bmatrix} + t \begin{bmatrix}[r]
            0 \\
            0 \\
            -1 \\
            1 
        \end{bmatrix} = \begin{cases}{}
            \textbf{u}_3 = \left( -\ddfrac{1}{\sqrt{2}} , \ddfrac{1}{\sqrt{2}} , 0, 0 \right) \\
            \textbf{u}_4 = \left( 0, 0, -\ddfrac{1}{\sqrt{2}} , \ddfrac{1}{\sqrt{2}}  \right) 
        \end{cases}$
        \begin{equation*}
            \begin{split}
                P &= \begin{bmatrix}[rrrr]
                    \ddfrac{1}{\sqrt{2}}  & 0 & -\ddfrac{1}{\sqrt{2}}  & 0 \\
                    \ddfrac{1}{\sqrt{2}}  & 0 & \ddfrac{1}{\sqrt{2}}  & 0 \\
                    0 & \ddfrac{1}{\sqrt{2}}  & 0 & -\ddfrac{1}{\sqrt{2}}  \\
                    0 & \ddfrac{1}{\sqrt{2}}  & 0 & \ddfrac{1}{\sqrt{2}} 
                \end{bmatrix}  \\
                    P^TAP &= \begin{bmatrix}[rrrr]
                        2 & 0 & 0 & 0 \\
                        0 & 2 & 0 & 0 \\
                        0 & 0 & 0 & 0 \\
                        0 & 0 & 0 & 0 
                    \end{bmatrix} 
            \end{split}
        \end{equation*}
 

%#################################################################################################
    \pagebreak

    \begin{Ques}
    \[ A = \begin{bmatrix}[rrr]
        1 & a & 0 \\
        a & 1 & 0 \\
        0 & 0 & -3 
    \end{bmatrix} , \quad a \in \R \]

    
    \end{Ques}


    (a) Find the eigenvalues of $A$, then determine conditions on $a$ such that $A$ has 3 distinct eigenvalues.

    \textit{\textcolor{blue5}{SOLUTION.}}  The characteristic polynomial of $A$ is
    \begin{equation*}
        \begin{split}
            | \lambda I - A| &= \begin{vmatrix}[rrr]
                \lambda  - 1 & -a & 0\\
                -a & \lambda  - 1 & 0 \\
                0 & 0 & \lambda  + 3 
            \end{vmatrix} = ( \lambda + 3 ) \begin{vmatrix}[rr]
                \lambda  - 1 & -a \\
                -a & \lambda - 1
            \end{vmatrix} \\
                             &= ( \lambda + 3 )[\lambda  - (a + 1)][\lambda - (-a + 1)]
        \end{split}
    \end{equation*}
    $A$ has 3 distinct eigenvalues $\Leftrightarrow \begin{cases}{}
        a + 1 \ne -3 \\
        -a + 1 \ne -3 \\
        a + 1 \ne -a + 1
    \end{cases} \implies \begin{cases}{}
        a \ne \pm 4 \\
        a \ne 0
    \end{cases}$

    (b) Let $a = 2$. Find an orthogonal matrix $P$ such that $P^TAP$ is diagonal and find that diagonal matrix.
    
    \textit{\textcolor{blue5}{SOLUTION.}}  Substuting $a = 2$ into the characteristic polynomial, obtain $ \begin{cases}{}
        \lambda _1 = -3 \\
        \lambda _2 = 3 \\
        \lambda _3 = -1
    \end{cases}$

    $\bullet$ For $\lambda _1 = -3$:

    $-3I - A = \begin{bmatrix}[rrr]
        -4 & -2 & 0 \\
        -2 & -4 & 0 \\
        0 & 0 & 0 
    \end{bmatrix} \to \begin{bmatrix}[ccc]
        2 & 1 & 0 \\
        0 & 3 & 0 \\
        0 & 0 & 0 
    \end{bmatrix} $. Let $s = x_3$, then $ \textbf{x} = \begin{bmatrix}[r]
        0 \\ 0 \\ t 
    \end{bmatrix} = t \begin{bmatrix}[r]
        0 \\ 0 \\ 1 
    \end{bmatrix} $$ \implies \textbf{u}_1 = (0,0,1)$

    $\bullet$ For $ \lambda_2 = 3$: 
   
    $3I - A = \begin{bmatrix}[rrr]
        2 & -2 & 0 \\
        -2 & 2 & 0 \\
        0 & 0 & 6 
    \end{bmatrix} \to \begin{bmatrix}[rrr]
        1 & -1 & 0 \\
        0 & 0 & 1 \\
        0 & 0 & 0 
    \end{bmatrix} $. Let $s = x_2$, then $ \textbf{x} = \begin{bmatrix}[r]
        s \\
        s \\
        0 
    \end{bmatrix}  = s \begin{bmatrix}[r]
        1 \\ 
        1 \\
        0 
    \end{bmatrix}  \implies \textbf{u}_2 = \ddfrac{1}{\sqrt{2}} (1,1,0) = \left( \ddfrac{1}{\sqrt{2}} , 
    \ddfrac{1}{\sqrt{2}} , 0\right) $

    $\bullet$ For $ \lambda _3 = -1$:

    $ \lambda I - A = \begin{bmatrix}[rrr]
        -2 & -2 & 0 \\
        -2 & -2 & 0 \\
        0 & 0 & 2 
    \end{bmatrix} \to \begin{bmatrix}[rrr]
        1 & 1 & 0 \\
        0 & 0 & 1 \\
        0 & 0 & 0 
    \end{bmatrix}.$ Let $s = x_2$, then $ \textbf{x} = \begin{bmatrix}[r]
        -s \\
        s \\
        0 
    \end{bmatrix} = s \begin{bmatrix}[r]
        -1 \\
        1 \\ 
        0 
    \end{bmatrix} \implies \textbf{u}_3 = \ddfrac{1}{\sqrt{2}} (-1,1,0) = \left( - \ddfrac{1}{\sqrt{2}} ,
    \ddfrac{1}{\sqrt{2}} , 0\right)$ 
     \begin{equation*}
        \begin{split}
            P &= [ \textbf{u}_1\text{ } \vdots \text{ }\textbf{u}_2 \text{  } \vdots \text{  }\textbf{u}_3 ] =  \begin{bmatrix}[rrr]
        0 & \frac{1}{\sqrt{2}} & - \frac{1}{\sqrt{2}} \\
        0 & \frac{1}{\sqrt{2}}& \frac{1}{\sqrt{2}} \\
        1 & 0 & 0 
    \end{bmatrix} \\
                P^TAP &= \begin{bmatrix}[ccc]
                    \lambda _1 & 0 & 0 \\
                    0 & \lambda _2 & 0 \\
                    0 & 0 & \lambda _3 
                \end{bmatrix}  =  \begin{bmatrix}[rrr]
                    -3 & 0 & 0 \\
                    0 & 3 & 0 \\
                    0 & 0 & -1 
                \end{bmatrix} 
        \end{split}
    \end{equation*}
\end{document}
