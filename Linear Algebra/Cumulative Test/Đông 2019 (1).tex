\documentclass{article}
\usepackage[utf8]{inputenc, vietnam}
\usepackage{amsmath, amssymb, systeme, mathtools, lmodern, float, graphicx}
\usepackage[most]{tcolorbox}
\usepackage[scale=.95,type1]{cabin}
\usepackage[framemethod=tikz]{mdframed}

\usepackage[legalpaper,margin=0.4in]{geometry}

\usepackage[nodisplayskipstretch]{setspace}


\setlength{\parindent}{10pt}
\setlength{\parskip}{1em}
\renewcommand{\baselinestretch}{1.2}

\title{Đề thi kết thúc môn học, Đông 2019 (1)}
\author{Trần Thùy Dung}
\date{}

\makeatletter
\renewcommand*\env@matrix[1][*\c@MaxMatrixCols c]{%
  \hskip -\arraycolsep
  \let\@ifnextchar\new@ifnextchar
  \array{#1}}
\makeatother

\newcommand\y{\cellcolor{blue!10}}

\usepackage{tabularray}

\newcommand\R{\mathbb{R}}

\DeclarePairedDelimiter\abs{\lvert}{\rvert}%
\DeclarePairedDelimiter\norm{\lVert}{\rVert}%

% Swap the definition of \abs* and \norm*, so that \abs
% and \norm resizes the size of the brackets, and the 
% starred version does not.
\makeatletter
\let\oldabs\abs
\def\abs{\@ifstar{\oldabs}{\oldabs*}}
%
\let\oldnorm\norm
\def\norm{\@ifstar{\oldnorm}{\oldnorm*}}
\makeatother

\newcommand\ddfrac[2]{\frac{\displaystyle #1}{\displaystyle #2}}

\syscodeextracol{\quad\hfill}{\hfill}
%\sysautonum*{(\uppercase\expandafter{\romannumeral*})}
\sysdelim..


\begin{document}
    \maketitle

    \begin{minipage}[t]{0.45\linewidth}
    \textbf{1.}

    $\quad \begin{bmatrix}[rrrr|r]
        1  & -4 & 3 & 2 & 5 \\
        0 & -15 & 11 & 4 & 8 \\
        0 & 0 & 5m+6 & -5m-6 & -22
    \end{bmatrix} $

    (a) Thay $m = 1$, ta có:
    $ \begin{bmatrix}[rrrr|r]
        1 & 0 & 0 & 1 & 3 \\
        0 & 1 & 0 & -1  & -2 \\
        0 & 0 & 1 & -1 & -2 
    \end{bmatrix} $

    Hệ phương trình có vô số nghiệm dạng $ \begin{bmatrix}[r]
        3 - t \\
        -2 + t \\
        -2 + t \\
        t 
    \end{bmatrix} $

    (b) 

    $ \begin{array}{l}
     m = - \ddfrac{6}{5} , \quad \text {hệ vô nghiệm}. \\
        m \ne - \ddfrac{6}{5} ,\quad \text{hệ có vô số nghiệm.} 
    \end{array}$



        
    \end{minipage}
    \begin{minipage}[t]{0.35\linewidth}
    \textbf{2.}

    (a) 
    \begin{equation*}
        \begin{split}
            |A| &=  \begin{vmatrix}[rrrr]
                1 & 2 & 3 & 4 \\
                -1 & 0 & 3  & 4 \\
                -1 & -2 & 0 & 4 \\
                -1 & -2 & -3 & 0 
            \end{vmatrix} =
            \begin{vmatrix}[rrrr]
                1 & 2 & 3 & 4 \\
                0 & 2 & 3 & 0 \\
                0 & 0 & 3 & 4 \\
                0 & 0 & 0 & 4
            \end{vmatrix} \\
            &= 1.2.3.4 = 24
        \end{split}
    \end{equation*}


    (b)
    Vì $|A| \ne 0 \implies$ $A$ khả nghịch.
    \[A^{-1} = \begin{bmatrix}[rrrr]
        0 & -1 & 1 & -1 \\
        1/2 & 1/2 & -1 & 1 \\
        -1/3 & 0 & 1/3 & -2/3 \\
        1/4 & 0 & 0 & 1/4 
    \end{bmatrix} \]

        
    \end{minipage}

    \begin{minipage}[t]{0.44\linewidth}
    \textbf{3.}

    (a) Ma trận chuẩn tắc của $T$ là $A = \begin{bmatrix}[rrr]
        1 & 0 & -1 \\
        2 & -1 & -2 \\
        -1 & 2 & 1 
    \end{bmatrix} $

    (b) Thực hiện các biến đổi sơ cấp trên $A$ ta được
    \[ \begin{bmatrix}[rrr]
        1 & 0 & -1 \\
        0 & -1 & 0 \\
        0 & 0 & 0 
    \end{bmatrix} \]

    Không gian hạch ker($T$) là không gian nghiệm của $A$:  $N(A)$.
    Cơ sở ker($T$) là $\text{span} \left\{ \begin{bmatrix}[r]
        1 \\
        0 \\
        1 
    \end{bmatrix}  \right\}$

    (c) Không, vì  (0,-1,1) không phải là tổ hợp tuyến tính của 2 cột đầu của $A$.

        
    \end{minipage} \hfill
    \begin{minipage}[t]{0.5\linewidth}
    \textbf{4.}

    (a)
    \setstretch{1.5}
    $ \begin{array}{l}
        d_{12}^2 = a^2 - 2a + 2 \\
        d_{23}^2 = 2a^2 + 2 \\
        d_{31}^2 = a^2 + 2a + 1
    \end{array}$

    $ d_{12} = d_{31} \Leftrightarrow a = \ddfrac{1}{4} $ 

    $ d_{12} = d_{23} \Leftrightarrow a^2 + 2a = 0 $

    Suy ra, không có $a$ thỏa mãn điều kiện.

    (b)  
   \begin{equation*}
        \begin{split}
            \textbf{u}_1 &= \left( \ddfrac{1}{\sqrt{2}} , 0, \ddfrac{1}{\sqrt{2}}  \right) \\
            \textbf{u}_2 &= (0, 1, 0) \\
            \textbf{u}_3 &= \left( \ddfrac{1}{\sqrt{2}} , 0,  - \ddfrac{1}{\sqrt{2}}  \right)
        \end{split}
    \end{equation*} 

        
    \end{minipage}





    \textbf{5.}

    (a) Đa thức đặc trưng của $A$ là 
    \[| \lambda I - A | = \begin{vmatrix}[rrr]
        \lambda  & 3a & 0 \\
        -1 & \lambda  - 2a & 0 \\
        -1 & 3 & \lambda  - 1 
    \end{vmatrix} = ( \lambda  - 1 )( \lambda ^2 - 2a \lambda  + 3a ) \]

    Phương trình luôn có nghiệm $ \lambda _1 = 1$   , do đó với mọi $a$ ta luôn có 
    $ \lambda = 1$ là giá trị riêng của $A$.

    \begin{minipage}{0.45\linewidth}
    (b) Thay $a = -1$, ta nhận được $ \begin{cases}{}
        \lambda _1 = 1 \quad \text{(bội 2)} \\
        \lambda _2 = -3
    \end{cases}$

    $\bullet$ Với $ \lambda _1 = 1$:

    $1I - A = \begin{bmatrix}[rrr]
        1 & -3 & 0 \\
        -1 & 3 & 0 \\
        -1 & 3 & 0 
    \end{bmatrix} \to \begin{bmatrix}[rrr]
        1 & -3 & 0 \\
        0 & 0 & 0 \\
        0 & 0 & 0 
    \end{bmatrix} $

    Các vector riêng:$(3,1,0), (0,0,1)$

    $\bullet$ Với $ \lambda _2 = -3$:

    $-3I - A = \begin{bmatrix}[rrr]
        -3 & -3 & 0 \\
        -1 & -1 & 0 \\
        -1  & 3 & -4 
    \end{bmatrix} \to \begin{bmatrix}[rrr]
        1 & 0 & 1 \\
        0 & 1 & -1 \\
        0 & 0 & 0 
    \end{bmatrix} $

    Vector riêng: $(-1,1,1)$ 

        
    \end{minipage}
    \begin{minipage}{0.45\linewidth}
    Ma trận $P$ là
    \begin{equation*}
        \begin{split}
            P  &= \begin{bmatrix}[rrr]
        3 & 0 & -1 \\
        1 & 0 & 1 \\
        0 & 1 & 1 
    \end{bmatrix}  \\
        P^{-1}AP &= \begin{bmatrix}[rrr]
            1 & 0 & 0 \\
            0 & 1 & 0 \\
            0 & 0 & -3 
        \end{bmatrix} 
        \end{split}
    \end{equation*}

        
    \end{minipage}


\end{document}
