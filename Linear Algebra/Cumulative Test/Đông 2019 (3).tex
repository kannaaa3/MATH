\documentclass{article}
\usepackage[utf8]{inputenc, vietnam}
\usepackage{amsmath, amssymb, systeme, mathtools, lmodern, float, graphicx}
\usepackage[most]{tcolorbox}
\usepackage[scale=.95,type1]{cabin}
\usepackage[framemethod=tikz]{mdframed}

\usepackage[legalpaper,margin=0.4in]{geometry}

\usepackage[nodisplayskipstretch]{setspace}


\setlength{\parindent}{10pt}
\setlength{\parskip}{1em}
\renewcommand{\baselinestretch}{1.2}

\title{Đề thi kết thúc môn học, Đông 2019 (3)}
\author{Trần Thùy Dung}
\date{}

\makeatletter
\renewcommand*\env@matrix[1][*\c@MaxMatrixCols c]{%
  \hskip -\arraycolsep
  \let\@ifnextchar\new@ifnextchar
  \array{#1}}
\makeatother

\newcommand\y{\cellcolor{blue!10}}

\usepackage{tabularray}
\SetTblrInner{colsep=5pt,rowsep=1pt}

\newcommand\x{\times}
\newcommand\xor{\oplus}

\makeatletter
\newcommand{\dashover}[2][\mathop]{#1{\mathpalette\df@over{{\dashfill}{#2}}}}
\newcommand{\fillover}[2][\mathop]{#1{\mathpalette\df@over{{\solidfill}{#2}}}}
\newcommand{\df@over}[2]{\df@@over#1#2}
\newcommand\df@@over[3]{%
  \vbox{
    \offinterlineskip
    \ialign{##\cr
      #2{#1}\cr
      \noalign{\kern1pt}
      $\m@th#1#3$\cr
    }
  }%
}
\newcommand{\dashfill}[1]{%
  \kern-.5pt
  \xleaders\hbox{\kern.5pt\vrule height.4pt width \dash@width{#1}\kern.5pt}\hfill
  \kern-.5pt
}
\newcommand{\dash@width}[1]{%
  \ifx#1\displaystyle
    2pt
  \else
    \ifx#1\textstyle
      1.5pt
    \else
      \ifx#1\scriptstyle
        1.25pt
      \else
        \ifx#1\scriptscriptstyle
          1pt
        \fi
      \fi
    \fi
  \fi
}
\newcommand{\solidfill}[1]{\leaders\hrule\hfill}
\makeatother

\newcommand\R{\mathbb{R}}

\DeclarePairedDelimiter\abs{\lvert}{\rvert}%
\DeclarePairedDelimiter\norm{\lVert}{\rVert}%

% Swap the definition of \abs* and \norm*, so that \abs
% and \norm resizes the size of the brackets, and the 
% starred version does not.
\makeatletter
\let\oldabs\abs
\def\abs{\@ifstar{\oldabs}{\oldabs*}}
%
\let\oldnorm\norm
\def\norm{\@ifstar{\oldnorm}{\oldnorm*}}
\makeatother

\newcommand*{\Value}{\frac{1}{2}x^2}%

\newcommand\ddfrac[2]{\frac{\displaystyle #1}{\displaystyle #2}}


\newcounter{Theo}[section]
\newenvironment{Theo}[1][]{%
  \stepcounter{Lemma}%
  \ifstrempty{#1}%
  {\mdfsetup{%
    frametitle={%
      \tikz[baseline=(current bounding box.east),outer sep=0pt]
      \node[line width=1pt,anchor=east,rectangle,draw=blue!20,fill=white]
    {\strut \color{black}{\textit{THEOREM}}~};}}
  }%
  {\mdfsetup{%
    frametitle={%
      \tikz[baseline=(current bounding box.east),outer sep=0pt]
      \node[line width=1pt,anchor=east,rectangle,draw=blue!20,fill=white]
    {\strut \color{black}{\textit{THEOREM}}~:~\color{blue5}{#1}};}}%
  }%
  \mdfsetup{innertopmargin=10pt,linecolor=blue!20,%
            linewidth=1pt,topline=true,%
            frametitleaboveskip=\dimexpr-\ht\strutbox\relax,}
  \begin{mdframed}[]\relax%
  }{\end{mdframed}}

\newcounter{Ques}[section]
\newenvironment{Ques}[1][]{%
    \stepcounter{Ques}
  \ifstrempty{#1}%
  {\mdfsetup{%
    frametitle={%
      \tikz[baseline=(current bounding box.east),outer sep=0pt]
      \node[line width=1pt,anchor=east,rectangle,draw=blue!20,fill=white]
    {\strut \color{blue5}{\textit{QUESTION}}~\theQues};}}
  }%
  {\mdfsetup{%
    frametitle={%
      \tikz[baseline=(current bounding box.east),outer sep=0pt]
      \node[line width=1pt,anchor=east,rectangle,draw=blue!20,fill=white]
    {\strut \color{black}{\textit{ANSWER}}~:~\color{blue4}{#1}};}}%
  }%
  \mdfsetup{innertopmargin=2pt,linecolor=blue!20,%
            linewidth=1pt,topline=true,%
            frametitleaboveskip=\dimexpr-\ht\strutbox\relax,}
  \begin{mdframed}[]\relax%
  }{\end{mdframed}}

\syscodeextracol{\quad\hfill}{\hfill}
%\sysautonum*{(\uppercase\expandafter{\romannumeral*})}
\sysdelim..


\begin{document}
    \maketitle
    \textbf{1.} 

    \begin{minipage}{0.30\linewidth}
    (a) $ \begin{cases}{}
        x = 1 \\
        y = -2 \\
        z = -2
    \end{cases} \quad $ 
    
    \end{minipage} 
    \begin{minipage}{0.60\linewidth}
    (b) $ \begin{bmatrix}[rrr|r]
        1 & 3 & 0 & -5 \\
        0 & 1 & 2 & -6 \\
        0 & 0 & 1 - m & m 
    \end{bmatrix} \implies \begin{array}{l}
        m = 1 : \text{ hệ vô nghiệm } \\
        m \ne 1 : \text{hệ có nghiệm duy nhất}
    \end{array}$

    
    \end{minipage}




    \textbf{2.}

    \begin{minipage}{0.45\linewidth}
    (a) $I_3 - 2A^TA = \begin{bmatrix}[rrr]
        8/9 & 0 & 0 \\
        0 & 5/9 & 0 \\
        0 & 0 & 5/9 
    \end{bmatrix} $

    (b) $(I_3 - 2A^TA)^2 = \begin{bmatrix}[rrr]
        64/81 & 0 & 0 \\
        0 & 25/81 & 0 \\
        0 & 0 & 25/81 
    \end{bmatrix} $

    (c) det($I_3 - 2A^TA$) > 0 nên $I_3 - 2A^TA$ khả nghịch.
    \[(I_3 - 2A^TA)^{-1} = \begin{bmatrix}[rrr]
        81/64 & 0 & 0 \\
        0 & 81/25 & 0 \\
        0 & 0 & 81/25 
    \end{bmatrix} \]

        
    \end{minipage}

    
    \textbf{3.}

    (a) $ \begin{cases}{}
        T(v_1)   + T(v_2) = -v_1 - v_2 = T(v_1 + v_2) \\
        T(cv_1) = c(-v_1) = cT(v_1)
    \end{cases}$

    $\implies$ $T_1$ là ánh xạ tuyến tính.

    Ma trận chính tắc của $T$ là $ \begin{bmatrix}[r]
        -1 
    \end{bmatrix} $.

    (b) $ \begin{cases}{}
        T_2(cx,cy,cz) = -(cx + 1, cy + cz) \ne -(cx + c, cy + cz) = c[-(x+1,y+z)] = cT_2(x,y,z)
    \end{cases}$

    Vậy $T_2$ không là ánh xạ tuyến tính.

    (c) $ \begin{cases}{}
        T(x_1, y_1, z_1) + T(x_2, y_2, z_2) = (x_1 + x_2, 0, 0) = T(x_1 +x_2, y_1+y_2, z_1 + z_2) \\
        T(cx_1, cy_1, cz_1) = (cx_1, 0, 0) = cT(x_1, y_1, z_1)
    \end{cases}$

    $\implies$ $T_3$ là ánh xạ tuyến tính.

    Ma trận chính tắc của $T$ là $ \begin{bmatrix}[rrr]
        1 & 0 & 0 \\
        0 & 0 & 0 \\
        0 & 0 & 0 
    \end{bmatrix} $.

    \begin{minipage}[t]{0.45\linewidth}
    \textbf{4.} 

    \[A = \begin{bmatrix}[rrr]
        1 & 1 & 1 \\
        1 & a & 1 \\
        1 & 1 & a+1 
    \end{bmatrix} \]

    (a) Số chiều của không gian ảnh $\text{dim}(T) = \text{dim}(A)$.
    Dùng các biến đổi sơ cấp trên hàng của $A$ ta nhận được ma trận
    \[ \begin{bmatrix}[rrr]
        1 & 1 & 1 \\
        0 & a-1 & 0 \\
        0 & 0 & a 
    \end{bmatrix} \]
    Để $\text{dim}(A)=2 \implies a - 1 = 0$ hoặc $a = 0$.

    $\implies a = 0$ hoặc $a = 1$.

    (b) Chọn $a = 1$ ta nhận được ma trận 
    \[ \begin{bmatrix}[rrr]
        1 & 1 & 1 \\
        0 & 0 & 0 \\
        0 & 0 & 1 
    \end{bmatrix} \to \begin{bmatrix}[rrr]
        1 & 1 & 0 \\
        0 & 0 & 1 \\
        0 & 0 & 0 
    \end{bmatrix} \]

    Khi đó, không gian ảnh của $T$ gồm các vector là tổ hợp tuyến tính của cột 1 và cột 3 của $A$  .
    \[\begin{cases}{}
        \textbf{u}_1 = \ddfrac{1}{\sqrt{3}} (1,1,1) = \left( \ddfrac{1}{\sqrt{3}} , \ddfrac{1}{\sqrt{3}} , \ddfrac{1}{\sqrt{3}}  \right) \\
        \textbf{w} _2 = \textbf{v}_2 - ( \textbf{u}_1 \cdot \textbf{v}_2 ) \textbf{u}_1 = \left( - \ddfrac{1}{3} , - \ddfrac{1}{3} , \ddfrac{2}{3}  \right)\\ \implies \textbf{u}_2 = \left( - \ddfrac{1}{\sqrt{6}} , - \ddfrac{1}{\sqrt{6}} , \ddfrac{2}{\sqrt{6}}  \right)
    \end{cases} \]
    Vậy $B' = \left\{ \left( \ddfrac{1}{\sqrt{3}} , \ddfrac{1}{\sqrt{3}} , \ddfrac{1}{\sqrt{3}}  \right) , \left( - \ddfrac{1}{\sqrt{6}} , - \ddfrac{1}{\sqrt{6}} , \ddfrac{2}{\sqrt{6}}  \right) \right\} $.
    
        
    \end{minipage} \hfill
    \begin{minipage}[t]{0.48\linewidth}
    \textbf{5.}

    (a) 

    $| \lambda I - A| = \begin{vmatrix}[rrr]

        \lambda + 1 & -2 & 2 \\
        -2 & \lambda  + 1 & -2 \\
        2 & -2 & \lambda  - 4 
    \end{vmatrix} = ( \lambda + 4 )( \lambda  - 5 )( \lambda  - 1 )$
    
    Vậy $A$ có 3 giá trị riêng $\lambda _1 = -4, \lambda _2 = 5, \lambda _3 = 1$.

    (b)

    $\bullet$ Với $ \lambda _1 = -4:$  

    $-4I - A = \begin{bmatrix}[rrr]
        -3 & -2 & 2 \\
        -2  & -3 & -2  \\
        2 & -2 & -8
    \end{bmatrix} \to \begin{bmatrix}[rrr]
        1 & 0 & -2 \\
        0 & 1 & 2 \\
        0 & 0 & 0 
    \end{bmatrix} $
    
    $\implies \textbf{u}_1 = \left( \ddfrac{2}{\sqrt{5}} , - \ddfrac{2}{\sqrt{5}} , \ddfrac{1}{\sqrt{5}}  \right)$

    $\bullet$ Với $ \lambda _2 = 5: 5I - A = \begin{bmatrix}[rrr]
        6 & -2 & 2 \\
        -2 & 6 & -2 \\
        2 & -2 & 1 
    \end{bmatrix} \to \begin{bmatrix}[rrr]
        4 & 0 & 1 \\
        0 & 4 & -1 \\
        0 & 0 & 0 
    \end{bmatrix} $
    
    $\implies \textbf{u}_2 = \left( - \ddfrac{1}{3\sqrt{2}} , \ddfrac{1}{3\sqrt{2}} , \ddfrac{4}{3\sqrt{2}}  \right)$

    $\bullet$ Với $ \lambda _3 = 1: 1I - A = \begin{bmatrix}[rrr]
        2 & -2 & 2 \\
        -2 & 2 & -2 \\
        2 & -2 & -3 
    \end{bmatrix} \to \begin{bmatrix}[rrr]
        1 & -1 & 0 \\
        0 & 0 & 1 \\
        0 & 0 & 0 
    \end{bmatrix} $

    $\implies \textbf{u}_3 = \left( \ddfrac{1}{\sqrt{2}} ,  \ddfrac{1}{\sqrt{2}}, 0 \right)$

    Vậy ma trận $P$ là 
    \begin{equation*}
        \begin{split}
            P &= \begin{bmatrix}[rrr]
                \ddfrac{2}{\sqrt{5}} & - \ddfrac{1}{3\sqrt{2}} & \ddfrac{1}{\sqrt{2}}  \\
                - \ddfrac{2}{\sqrt{5}}  & \ddfrac{1}{3\sqrt{2}} & \ddfrac{1}{\sqrt{2}} \\
                \ddfrac{1}{\sqrt{5}} & \ddfrac{4}{3\sqrt{2}} & 0 
            \end{bmatrix} \\
                P^TAP &= \begin{bmatrix}[rrr]
                    -4 & 0 & 0 \\
                    0 & 5 & 0 \\
                    0 & 0 & 1 
                \end{bmatrix} 
        \end{split}
    \end{equation*}

    \end{minipage}


  \end{document}

