\documentclass{article}
\usepackage[utf8]{inputenc, vietnam}
\usepackage{amsmath, amssymb, systeme, mathtools, lmodern, float, graphicx}
\usepackage[most]{tcolorbox}
\usepackage[scale=.95,type1]{cabin}
\usepackage[framemethod=tikz]{mdframed}

\usepackage[legalpaper,margin=0.4in]{geometry}

\usepackage[nodisplayskipstretch]{setspace}


\setlength{\parindent}{10pt}
\setlength{\parskip}{1em}
\renewcommand{\baselinestretch}{1.2}

\title{Đề thi kết thúc môn học, Đông 2019 (2)}
\author{Trần Thùy Dung}
\date{}

\makeatletter
\renewcommand*\env@matrix[1][*\c@MaxMatrixCols c]{%
  \hskip -\arraycolsep
  \let\@ifnextchar\new@ifnextchar
  \array{#1}}
\makeatother

\newcommand\y{\cellcolor{blue!10}}

\usepackage{tabularray}

\newcommand\R{\mathbb{R}}

\DeclarePairedDelimiter\abs{\lvert}{\rvert}%
\DeclarePairedDelimiter\norm{\lVert}{\rVert}%

% Swap the definition of \abs* and \norm*, so that \abs
% and \norm resizes the size of the brackets, and the 
% starred version does not.
\makeatletter
\let\oldabs\abs
\def\abs{\@ifstar{\oldabs}{\oldabs*}}
%
\let\oldnorm\norm
\def\norm{\@ifstar{\oldnorm}{\oldnorm*}}
\makeatother

\newcommand\ddfrac[2]{\frac{\displaystyle #1}{\displaystyle #2}}

\syscodeextracol{\quad\hfill}{\hfill}
%\sysautonum*{(\uppercase\expandafter{\romannumeral*})}
\sysdelim..


\begin{document}
    \maketitle

    \begin{minipage}[t]{0.45\linewidth}
    \textbf{1.} 

    $ \begin{bmatrix}[rrrr|r]
        -3 & 3 & -4 & 11 & 6 \\
        0 & -9 & -13 & 8 & 0 \\
        0 & 0  & 27a - 38 & -27a^2 + 49 & 0 
    \end{bmatrix} $

    (a) Với $a = 1$: $ \begin{bmatrix}[rrrr|r]
        1 & 0 & 0 & 1 & -2 \\
        0 & 1 & 0 & 2 & 0 \\
        0 & 0 & 1  & -2 & 0 
    \end{bmatrix} $

    $\implies$ Hệ có vô số nghiệm dạng $  \begin{bmatrix}[r]
        -2 \\ 0 \\ 0 \\ 0 
    \end{bmatrix} + t \begin{bmatrix}[r]
        -1 \\
        -2 \\
        2 \\
        1 
    \end{bmatrix} $

    (b) Hệ phương trình vô số nghiệm với mọi $a$.

        
    \end{minipage} \hfill
    \begin{minipage}[t]{0.48\linewidth}
    \textbf{2.}
    
    (a) $A  + B = \begin{bmatrix}[rrr]
        1 & 1 & 1 \\
        2 & -1 & 1 \\
        3 & -2 & 2 
    \end{bmatrix} $

    $ \begin{vmatrix}[rrr]
        1 & 1 & 1 \\
        2 & -1 & 1 \\
        3 & -2 & 2 
    \end{vmatrix} \xrightarrow[C_2 + C_3 \to C_2]{C_1 - C_3 \to C_1} \begin{vmatrix}[rrr]
        0 & 2 & 1 \\
        1 & 0 & 1 \\
        1 & 0 & 2 
    \end{vmatrix} = -2 \begin{vmatrix}[rr]
        1 & 1 \\
        1 & 2 
    \end{vmatrix} = -2$

    Ta có det($A + B$) $\ne 0$, do đó $A + B$ có nghịch đảo.

    (b) $ \begin{bmatrix}[rrr|r]
        1 & 1 & 1 & 4 \\
        2 & -1 & 1 & 6 \\
        3 & -2 & 2 & 0 
    \end{bmatrix} \to \begin{bmatrix}[rrr|r]
        1 & 0 & 0 & 12 \\
        0 & 1 & 0 & 5 \\
        0 & 0 & 1 & -13 
    \end{bmatrix} $

    Vậy phương trình có nghiệm duy nhất $X = \begin{bmatrix}[r]
        12 \\
        5 \\
        -13 
    \end{bmatrix} $.

        
    \end{minipage}

    \begin{minipage}[t]{0.45\linewidth}
    \textbf{3.}

    (a) Ma trận chính tắc của $T$ là 
    \[A = \begin{bmatrix}[rrr]
        1 & 0 & 2 \\
        1 & -1 & 2 \\
        2 & 1 &  4 
    \end{bmatrix} \]

    (b) Thực hiện các biến đổi sơ cấp trên $A$ ta nhận được ma trận 
    \[ B = \begin{bmatrix}[rrr]
        1 & 0 & 2 \\
        0 & 1 & 0 \\
        0 & 0 & 0 
    \end{bmatrix} \]

    Không gian nghiệm của ma trận trên là $ \text{ker}(T) = \text{span} \left\{ \begin{bmatrix}[r]
        -2 \\
        0 \\
        1 
    \end{bmatrix}  \right\}$

    (c) Từ ma trận $B$, suy ra số chiều của không gian ảnh $\text{im}(T) = 2$.

    Cơ sở không gian ảnh gồm các vector cột 1 và 2 của $B$: im($T$) = $\text{span} \left\{ \begin{bmatrix}[r]
            1 \\ 0 \\ 0 
    \end{bmatrix} , \begin{bmatrix}[r]
    0  \\ 1 \\ 1 
    \end{bmatrix}  \right\}$

    Do số chiều của $\text{im}(T) < 3$, $T$ không phải là toàn cấu.

        
    \end{minipage} \hfill
    \begin{minipage}[t]{0.48\linewidth}
    \textbf{4.}

    (a) 
    \[ \begin{array}{cl}
        \textbf{v}_3 \cdot \textbf{v}_1 &= a + 3\\
        \textbf{v}_1 \cdot \textbf{v}_2 &= a + 4 \\
        \textbf{v}_2 \cdot \textbf{v}_3 &= 7 

    \end{array} \]

    Để 3 giá trị trên tạo thành cấp số cộng với công sai bằng 1 
    \[ \bigg[\begin{array}{l}
        a + 3 = 7 + 1 \\
        a + 4 = 7 - 1 
    \end{array} \implies \bigg[ \begin{array}{l}
        a = 5 \\
        a = 2 
    \end{array} \]

    (b) 
    \begin{equation*}
        \begin{split}
            \textbf{u}_1 &= \left( 0, \ddfrac{1}{\sqrt{2}} , \ddfrac{1}{\sqrt{2}}  \right) \\
            \textbf{u}_2 &= (1,0,0) \\
            \textbf{u}_3 &= \left( 0, \ddfrac{1}{\sqrt{2}} , - \ddfrac{1}{\sqrt{2}}  \right)
        \end{split}
    \end{equation*}


    \end{minipage}

    {\color{blue9} \rule{18cm}{0.3mm}}
\pagebreak
    

    \textbf{5.}

    \begin{minipage}{0.48\linewidth}
    (a)

    $| \lambda I - A| = \begin{vmatrix}[rrr]
        \lambda - 2 & -2 & -1\\
        -2 & \lambda  - 2 & 1 \\
        -1 & 1 & \lambda  + 1 
    \end{vmatrix}  = ( \lambda  + 2 )( \lambda  - 4 )( \lambda - 1 )$

    $\bullet$ Với $ \lambda _1 = -2$:

    $-2I - A = \begin{bmatrix}[rrr]
        -4 & -2 & -1 \\
        -2 & -4 & 1 \\
        -1 & 1 & -1 
    \end{bmatrix} \to \begin{bmatrix}[rrr]
        2 & 0 & 1 \\
        0 & 2 & -1 \\
        0 & 0 & 0 
    \end{bmatrix}$

    Không gian riêng: $ \text{span} \left\{ \begin{bmatrix}[r]
        -1/2 \\
        1/2 \\
        1 
    \end{bmatrix}  \right\} $

    $\bullet$  Với $ \lambda _2 = 4$:

    $4I - A = \begin{bmatrix}[rrr]
        2 & -2 & -1 \\
        -2 & 2 & 1 \\
        -1 & 1 &  5
    \end{bmatrix} \to \begin{bmatrix}[rrr]
        1 & -1 & 0 \\
        0 & 0 & 1 \\
        0 & 0 & 0 
    \end{bmatrix} $

    Không gian riêng: $\text{span} \left\{ \begin{bmatrix}[r]
        1 \\
        1\\
        0 
    \end{bmatrix}  \right\}$


    \end{minipage} \hfill
    \begin{minipage}{0.48\linewidth}

    $\bullet$ Với $ \lambda _3 = 1$:

    $ 1I - A = \begin{bmatrix}[rrr]
        -1 & -2 & -1 \\
        -2 & -1 & 1 \\
        -1 & 1 & 2 
    \end{bmatrix} \to \begin{bmatrix}[rrr]
        1 & 0 & -1 \\
        0 & 1 & 1 \\
        0 & 0 & 0 
    \end{bmatrix} $

    Không gian riêng: $\text{span} \left\{ \begin{bmatrix}[r]
        1 \\
        -1 \\
        1 
    \end{bmatrix}  \right\}$

    (b) Trực chuẩn hóa các vector riêng của $A$ ta nhận được ma trận $P$:
    \begin{equation*}
        \begin{split}
            P &= \begin{bmatrix}[rrr]
                -\ddfrac{1}{\sqrt{6}} & \ddfrac{1}{\sqrt{2}} & \ddfrac{1}{\sqrt{3}} \\
                \ddfrac{1}{\sqrt{6}} & \ddfrac{1}{\sqrt{2}} & - \ddfrac{1}{\sqrt{3}} \\
                \ddfrac{2}{\sqrt{6}} & 0 & \ddfrac{1}{\sqrt{3}} 
            \end{bmatrix} \\
                P^TAP &= \begin{bmatrix}[rrr]
                    -2 & 0 & 0 \\
                    0 & 4 & 0 \\
                    0 & 0 & 1 
                \end{bmatrix} 
        \end{split}
    \end{equation*}

        
    \end{minipage}
    
\end{document}
