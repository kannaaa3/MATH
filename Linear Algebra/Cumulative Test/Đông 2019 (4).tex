\documentclass{article}
\usepackage[utf8]{inputenc, vietnam}
\usepackage{amsmath, amssymb, systeme, mathtools, lmodern, float, graphicx}
\usepackage[most]{tcolorbox}
\usepackage[scale=.95,type1]{cabin}
\usepackage[framemethod=tikz]{mdframed}

\usepackage[legalpaper,margin=1in]{geometry}

\usepackage[nodisplayskipstretch]{setspace}


\setlength{\parindent}{10pt}
\setlength{\parskip}{1em}
\renewcommand{\baselinestretch}{1.2}

\title{Đề thi kết thúc môn học, Đông 2019 (4)}
\author{Trần Thùy Dung}
\date{}

\makeatletter
\renewcommand*\env@matrix[1][*\c@MaxMatrixCols c]{%
  \hskip -\arraycolsep
  \let\@ifnextchar\new@ifnextchar
  \array{#1}}
\makeatother

\newcommand\y{\cellcolor{blue!10}}

\usepackage{tabularray}
\SetTblrInner{colsep=5pt,rowsep=1pt}

\newcommand\x{\times}
\newcommand\xor{\oplus}

\makeatletter
\newcommand{\dashover}[2][\mathop]{#1{\mathpalette\df@over{{\dashfill}{#2}}}}
\newcommand{\fillover}[2][\mathop]{#1{\mathpalette\df@over{{\solidfill}{#2}}}}
\newcommand{\df@over}[2]{\df@@over#1#2}
\newcommand\df@@over[3]{%
  \vbox{
    \offinterlineskip
    \ialign{##\cr
      #2{#1}\cr
      \noalign{\kern1pt}
      $\m@th#1#3$\cr
    }
  }%
}
\newcommand{\dashfill}[1]{%
  \kern-.5pt
  \xleaders\hbox{\kern.5pt\vrule height.4pt width \dash@width{#1}\kern.5pt}\hfill
  \kern-.5pt
}
\newcommand{\dash@width}[1]{%
  \ifx#1\displaystyle
    2pt
  \else
    \ifx#1\textstyle
      1.5pt
    \else
      \ifx#1\scriptstyle
        1.25pt
      \else
        \ifx#1\scriptscriptstyle
          1pt
        \fi
      \fi
    \fi
  \fi
}
\newcommand{\solidfill}[1]{\leaders\hrule\hfill}
\makeatother

\newcommand\R{\mathbb{R}}

\DeclarePairedDelimiter\abs{\lvert}{\rvert}%
\DeclarePairedDelimiter\norm{\lVert}{\rVert}%

% Swap the definition of \abs* and \norm*, so that \abs
% and \norm resizes the size of the brackets, and the 
% starred version does not.
\makeatletter
\let\oldabs\abs
\def\abs{\@ifstar{\oldabs}{\oldabs*}}
%
\let\oldnorm\norm
\def\norm{\@ifstar{\oldnorm}{\oldnorm*}}
\makeatother

\newcommand*{\Value}{\frac{1}{2}x^2}%

\newcommand\ddfrac[2]{\frac{\displaystyle #1}{\displaystyle #2}}


\newcounter{Theo}[section]
\newenvironment{Theo}[1][]{%
  \stepcounter{Lemma}%
  \ifstrempty{#1}%
  {\mdfsetup{%
    frametitle={%
      \tikz[baseline=(current bounding box.east),outer sep=0pt]
      \node[line width=1pt,anchor=east,rectangle,draw=blue!20,fill=white]
    {\strut \color{black}{\textit{THEOREM}}~};}}
  }%
  {\mdfsetup{%
    frametitle={%
      \tikz[baseline=(current bounding box.east),outer sep=0pt]
      \node[line width=1pt,anchor=east,rectangle,draw=blue!20,fill=white]
    {\strut \color{black}{\textit{THEOREM}}~:~\color{blue5}{#1}};}}%
  }%
  \mdfsetup{innertopmargin=10pt,linecolor=blue!20,%
            linewidth=1pt,topline=true,%
            frametitleaboveskip=\dimexpr-\ht\strutbox\relax,}
  \begin{mdframed}[]\relax%
  }{\end{mdframed}}

\newcounter{Ques}[section]
\newenvironment{Ques}[1][]{%
    \stepcounter{Ques}
  \ifstrempty{#1}%
  {\mdfsetup{%
    frametitle={%
      \tikz[baseline=(current bounding box.east),outer sep=0pt]
      \node[line width=1pt,anchor=east,rectangle,draw=blue!20,fill=white]
    {\strut \color{blue5}{\textit{QUESTION}}~\theQues};}}
  }%
  {\mdfsetup{%
    frametitle={%
      \tikz[baseline=(current bounding box.east),outer sep=0pt]
      \node[line width=1pt,anchor=east,rectangle,draw=blue!20,fill=white]
    {\strut \color{black}{\textit{ANSWER}}~:~\color{blue4}{#1}};}}%
  }%
  \mdfsetup{innertopmargin=2pt,linecolor=blue!20,%
            linewidth=1pt,topline=true,%
            frametitleaboveskip=\dimexpr-\ht\strutbox\relax,}
  \begin{mdframed}[]\relax%
  }{\end{mdframed}}

\syscodeextracol{\quad\hfill}{\hfill}
%\sysautonum*{(\uppercase\expandafter{\romannumeral*})}
\sysdelim..

\begin{document}
\iffalse
   \begin{Ques}
       \[ \systeme[xyz]{
               2x + 3y + z= 10,
               mx - 3y - 3z = 22,
               4x - 2y + 3z = -2
           }\]
   \textbf{a.} Solve the system with $m = 2$.\\ 
   \textbf{b.} chiu
    \end{Ques} 
\fi
    \maketitle

    \setstretch{1.5}
    \begin{minipage}{0.45\linewidth}

    \textbf{1.} (a) $ \begin{cases}{}
        x = 5 \\ y = 2 \\ z = -6 
    \end{cases}$


    (b) $ \begin{array}{l}
        m \ne -\frac{54}{11} : \quad \text{hệ vô nghiệm}\\
        m = -\frac{54}{11} : \quad \text{hệ vô nghiệm}
    \end{array}$

        
\end{minipage} 
\setstretch{1.5}
\begin{minipage}{0.45\linewidth}
        \textbf{2.} 

    (a) $AB = \begin{bmatrix}[rrr]
        22 & 37 & 7 - 4m \\
        48 & 68 & 16 - 8m \\
        -30 + 2m & -60 - 3m & -9 + 7m 
    \end{bmatrix}$

    (b) Không, vì det($A$) = 0.
   

\end{minipage}

    \textbf{3.} 

    \textbf{(a)} Viết các vector của $B'$ thành các cột của ma trận $P$
    \[|P| = \begin{vmatrix}[rrr]
        2 & -1 & -2\\
        1 & 1 & 1 \\
        0 & 0 & 1 
    \end{vmatrix} = \begin{vmatrix}[rr]
        2 & -1 \\
        1 & 1 
    \end{vmatrix} = 3 \ne 0\]

    Suy ra, phương trình $c_1(2,1,0) + c_2(-1,1,0) + c_3(-2,1,1) = (0,0,0)$ có nghiệm duy nhất.

    $\implies$ $B'$ độc lập tuyến tính và có đúng $3$ vector, do đó $B'$ là cơ sở của $\R^3$.

    (b) Đối với cơ sở $B$ là cơ sở chính tắc của $ \R^3 $
    \[T_B = \begin{bmatrix}[rrr]
        1 & 16 & -12 \\
        2 & 5 & -2 \\
        0 & 0 & -1 
    \end{bmatrix} \]

    Ma trận $P$ chính là ma trận chuyển cơ sở từ $B'$ sang $B$. 
    \[P^{-1} = \begin{bmatrix}[rrr]
        \frac{1}{3} & \frac{1}{3} & \frac{1}{3} \\
        - \frac{1}{3} & \frac{2}{3} & -\frac{4}{3} \\
        0 & 0 & 1 
    \end{bmatrix} \]

    Ma trận $T$ đối với cơ sở $B'$ là 
    \[T_{B'} = P^{-1}T_BP = \begin{bmatrix}[rrr]
        9 & 6 & 0 \\
        0 & -3 & 0 \\
        0 & 0 & -1 
    \end{bmatrix} \]

    \textbf{4.} 

    \setstretch{1.5}
    \begin{minipage}{0.5\linewidth}
    (a)
    $\begin{array}{cl}
            d_{12} &= \sqrt{2a^2 - 8a + 11} \\
            d_{23} &= \sqrt{a^2 - 4a + 5} \\
            d_{31} &= \sqrt{a^2 - 2a + 6}
        
    \end{array}$
    
    $\bullet$ $ d_{12} = d_{23} \Leftrightarrow a^2 - 4a + 6 = (a - 2)^2 + 2 = 0$

    $\implies$ Không có $a$ thỏa mãn.

    
    \end{minipage}
    \setstretch{1.5}
    \begin{minipage}{0.45\linewidth}
    (b) 

    $            \textbf{u}_1 = \ddfrac{ \textbf{v}_1 }{\norm{ \textbf{v}_1 }}  = \left( \ddfrac{1}{\sqrt{5}} , 0 , \ddfrac{2}{\sqrt{5}}  \right)$

    $            \textbf{w} _2 = \textbf{v}_2 - ( \textbf{u}_1 \cdot \textbf{v}_2 ) \textbf{u}_1 = (0,1,0) $

    $\implies \textbf{u}_2 = \ddfrac{ \textbf{w} _2  }{\norm{ \textbf{w} _2  }} = (0,1,0)$

    $ \textbf{w} _3 = \textbf{v}_3 - ( \textbf{u}_1 \cdot \textbf{v}_3 ) \textbf{u}_1 - ( \textbf{u}_2 \cdot \textbf{v}_3 ) \textbf{u}_2 = \left( - \ddfrac{6}{5} , 0, \ddfrac{3}{5}  \right)$

    $\implies \textbf{u}_3 = \left( - \ddfrac{2}{\sqrt{5}} , 0, \ddfrac{1}{\sqrt{5}}  \right) $

        
    \end{minipage}


    \textbf{5.}

    (a) $| \lambda I - A| = \begin{vmatrix}[rrr]
        \lambda  - 3 & -3 & -2 \\
        -1 & \lambda  - 1 & 2 \\
        1 & 3 & \lambda  
    \end{vmatrix} = ( \lambda  + 2 )( \lambda  - 2 )( \lambda  - 4 )$

    $\bullet$ Với $ \lambda _1 = -2: -2I - A = \begin{bmatrix}[rrr]
        -5 & -3 & -2 \\
        -1 & -3 & 2 \\
        1 & 3 & -2 
    \end{bmatrix} \to \begin{bmatrix}[rrr]
        1 & 0 & 1 \\
        0 & 1 & -1 \\
        0 & 0 & 0 
    \end{bmatrix} $

    $\implies$ Không gian con riêng: $\text{span} \left\{ \begin{bmatrix}[r]
        -1 \\
        1 \\
        1
    \end{bmatrix}  \right\}$

    $\bullet$  Với $ \lambda _2 = 2: 2I - A = \begin{bmatrix}[rrr]
        -1 & -3 & -2 \\
        -1 & 1 & 2 \\
        1 & 3 & 2 
    \end{bmatrix} \to \begin{bmatrix}[rrr]
        1 & 0 & -1 \\
        0 & 1 & 1 \\
        0 & 0 & 0 
    \end{bmatrix} $

   $\implies$ Không gian con riêng: $\text{span} \left\{ \begin{bmatrix}[r]
        1 \\ -1 \\ 1
    \end{bmatrix}  \right\}$

    $\bullet$  Với $ \lambda _3 = 4 : 4I - A = \begin{bmatrix}[rrr]
        1 & -3 & -2 \\
        -1 & 3 & 2 \\
        1 & 3 & 4 
    \end{bmatrix} \to \begin{bmatrix}[rrr]
        1 &  0 & 1\\
        0 & 1 & 1 \\
        0 & 0 & 0 
    \end{bmatrix} $

    Không gian con riêng: $\text:span \left\{ \begin{bmatrix}[r]
        -1 \\
        -1 \\
        1 
    \end{bmatrix}  \right\}$

    (b) 

    Viết các vector riêng của $A$ thành các cột của $P$
    \[P =  \begin{bmatrix}[rrr]
        -1 & 1 & -1 \\
        1 & -1 & -1 \\
        1 & 1 & 1 
    \end{bmatrix} \implies P^{-1} = \begin{bmatrix}[rrr]
        0 & 1/2 & 1/2 \\
        1/2 & 0 & 1/2 \\
        -1/2 & -1/2 & 0 
    \end{bmatrix} \]
    Ma trận đường chéo nhận được là 
    \[P^{-1}AP = \begin{bmatrix}[rrr]
        -2 & 0 & 0 \\
        0 & 2 & 0 \\
        0 & 0 & 4 
    \end{bmatrix} \]



\end{document}

